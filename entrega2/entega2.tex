%% 12pt, a4 paper, with title page
\documentclass[10pt, a4paper, titlepage,
% one-sided
	oneside,
% flush left equations, left equation numbers
	fleqn, leqno]{article}

\usepackage[top=2.5cm, bottom=2cm, left=2.5cm, right=1cm]{geometry}

% Spanish titles
\usepackage[spanish]{babel}

\usepackage{pdfpages}
%\usepackage{graphicx}
\usepackage{amsmath}
%\usepackage{gensymb}
\usepackage{multirow}
\usepackage{rotating}

% dot
%\usepackage{dot2texi}
%\usepackage{tikz}
%\usetikzlibrary{shapes,arrows}

% Font, encoding
\usepackage[utf8]{inputenc}
\usepackage[T1]{fontenc}
\usepackage{times}                % Otros: palatino, bookman, times
\usepackage{enumerate}
% Interactive index
\usepackage{hyperref}

% Attachments
\usepackage{attachfile}


% Make lists without bullets
%\renewenvironment{itemize}{
% \begin{list}{}{
%  \setlength{\leftmargin}{1.5em}
% }
%}{
% \end{list}
%}

% Open sections on right pages
\let\oldsection\section
\def\section{\cleardoublepage\oldsection}

\newcommand{\attachlisting}[1]{\textattachfile{#1}{#1}\lstinputlisting{#1}}

% Code listings
\usepackage{listings}
% Use a monospace font, don't pad with spaces
\lstset{basicstyle=\ttfamily,columns=flexible}

% Watermark
%\usepackage{draftwatermark}
%\SetWatermarkText{BORRADOR}

% Document properties
\title{71.14 - Modelos y Optimización I - TP1 - Segunda Entrega}
\author{
	Federico Churca-Torrusio \\\texttt{fchurca@fi.uba.ar}
	\and
	Cynthia Gamarra \\\texttt{cgamarra@fi.uba.ar}}
\date{2015 - Primer Cuatrimestre}
\hypersetup{
  colorlinks = true,
  urlcolor = red,
  pdflang = es,
  pdfauthor = {Federico Churca-Torrusio, Cynthia Gamarra},
  pdfproducer = {Federico Churca-Torrusio},
  pdfcreator = pdflatex,
  pdftitle = {71.14 - Modelos y Optimización I - TP1 - SEGUNDA ENTREGA},
  pdfsubject = {},
  pdfpagemode = UseNone
}

\begin{document}
\maketitle
%\thispagestyle{empty}
%\cleardoublepage
%\pagenumbering{roman}
%\setcounter{page}{1}
%\tableofcontents
%\cleardoublepage
%\listoffigures
%\listoftables

\cleardoublepage
\pagenumbering{arabic}
\setcounter{page}{1}

\section{Parte A}
\begin{enumerate} [a .]
\item \textbf{Plantear una heurística para resolver el problema. Tener en cuenta que la heurística debe ser genérica, es decir, debe servir no sólo para este problema sino para otros problemas similares.}\\
\begin{center}
\textbf{Seudocódigo de la heurística propuesta}
\end{center}
\begin{enumerate}
\item Se crea una lista de los sectores involucrados.
\item Para cada sector se le asigna una lista de condiciones a cumplir. Para cada condición se asigna 1 si el sector apoya a determinado conflicto, sino 0 (cero).
\item Se ordena de menor a mayor la lista de sectores involucrados que apoyen a la menor cantidad de conflictos.En caso de empate se dá mayor prioridad al sector que quiere al menos una victoria.
\item Mientras haya sectores:
\begin{enumerate}
\item Agregar al sector a la lista de SectoresApoyo como posible sector de apoyo al partido.En caso de empate agregar al sector que no pida vacantes en la lista.
\item Si se ofrece la vicepresidencia se agrega automáticamente al sector a la lista SectoresApoyo.
\item Fin mientras
\end{enumerate}
\item Si la longitud de la lista de SectoresApoyo coincide con la cantidad de sectores totales, se podrá ganar la interna.
\item Fin
\end{enumerate}

\item \textbf{Realizar un programa que ejecute la heurística planteada y analizar los resultados.}
\end{enumerate}


\section{Parte B}
A partir del resultado de la corrida del TP y del análisis de sensibilidad, responder las siguientes preguntas. Si consideran que les falta información, indicar qué información les falta y porqué. NO SE DEBE REALIZAR UNA NUEVA CORRIDA DEL TP ( ni por LINDO, ni por simplex). Tener en cuenta que lo que se espera de esta entrega es que sepan sacar el máximo provecho de la información disponible. Cada una de la preguntas es independiente de las demás.
\begin{enumerate} [a .]
\item \textbf{Se ha recalcalculado el beneficio de cada tren en funcionamiento, y se descubrió que cada tren produce un beneficio anual de \$ 4 millones. ¿Cómo cambia la solución obtenida con esta información?}\\
	Para ver esto, se analiza la sección de variación de coeficientes de objetivo.
Si vemos la fila \texttt{TRENES}, notamos que se permite desde un aumento de USD2M/U a un decremento de USD2.5M/U sobre el coeficiente de ganancia actual de USD2M/U\footnote{
	También podemos ver que el coeficiente de ganancia aparece como 0 por definir el funcional en una forma no canónica, maximizando una sola variable \texttt{INGRESOS}.
	Si cambiamos la definición del máximo por \texttt{max 2 Trenes + 10 ViasCubiertas} y quitamos la definición de \texttt{INGRESOS} obtenemos un modelo matemáticamente equivalente, y con la salida esperada de un coeficiente de ganancia de USD2M/U para \texttt{TRENES} y USD10M/Mm para \texttt{VIASCUBIERTAS}.}.
Como el incremento está en la frontera del rango de validez de la solución actual, puede realizarse sin cambiar las cantidades de trenes a comprar y vías a instalar.

	La ganancia nueva puede calcularse incrementando la actual con el producto del incremento del factor de ganancia para \texttt{TRENES} y la cantidad de \texttt{TRENES}:
\begin{align*}
	INGRESOS_a &= INGRESOS_i + \Delta C_{TRENES} * TRENES\\
		&= USD2600M + USD2M/U * 1040U\\
		&= USD2600M + USD2080M\\
	INGRESOS_a &= USD4680M
\end{align*}

\item \textbf{Para mejorar la imagen del partido, se quiere elevar el mínimo de trenes a inaugurar a 50. ¿Cómo afecta esto a la solución obtenida?}\\
	Analizamos esta vez la sección de variación de coeficientes de lado derecho.
En este caso, vemos que para la solución actual la cantidad de trenes mínima, aunque se puede reducir arbitrariamente, no se puede incrementar sin un cambio de solución.
Hay que calcular entonces cuánto del presupuesto erogado hay que redirigir de la compra de vías a la compra de trenes, calcular la cobertura nueva de vías, y calcular nuevamente los ingresos.
Como se está aumentando un mínimo, forzándonos a comprar un recurso que ya estaba en el mínimo original, lo más esperable es que la nueva ganancia sea peor.

	Empecemos por el presupuesto:
\begin{align*}
	PRESUPUESTO_{TRENES}	&= USD20M/U * 50U\\
	PRESUPUESTO_{TRENES}	&= USD1000M\\
	PRESUPUESTO_{VIAS}	&= PRESUPUESTO - PRESUPUESTO_{TRENES}\\
		&= USD1000M - USD1000M\\
	PRESUPUESTO_{VIAS}	&=0\\
\end{align*}
Vemos que sólamente hay presupuesto para la compra de trenes, y no se compran nuevas vías.

	Ahora analizamos la cobertura:
\begin{align*}
	VIASCUBIERTAS_b	&= min(48Mm, 1050U * 0.05Mm/U)\\
		&= min(48Mm, 52.5Mm)\\
	VIASCUBIERTAS_b	&= 48Mm
\end{align*}
Vemos que se cubre completamente la infraestructura vial actual.

	Ahora analicemos los ingresos:
\begin{align*}
	INGRESOS_b	&= USD2M/U * 1050U + USD10M/Mm * 48Mm\\
		&= USD2100M + USD480M\\
	INGRESOS_b	&= USD2580M
\end{align*}
Vemos que la solución de 2580 millones de USD es, efectivamente, peor que la original de 2600 millones de USD.

\item \textbf{Se presenta la posibilidad de incorporar nuevos trenes de alta velocidad. Cada uno cuesta 30 millones, cubre 100 kilómetros de vías y puede incluirse dentro de la cantidad mínima de 40 trenes. Cada tren de alta velocidad generará un beneficio de 3,8 millones anuales. ¿Será conveniente incorporar trenes de este tipo?}\\
	Esta vez analizamos la sección de valores marginales (dual prices) para calcular el lucro cesante requerido para incluir un nuevo tren de alta velocidad, y luego calcularemos ese lucro cesante con el beneficio de una unidad.
La diferencia entre ambos será el incremento o decremento, según el signo, de la ganancia obtenida en caso de usar una unidad.
Si es positiva, podría ser conveniente tomar al menos una unidad.

\begin{align*}
	LUCROCESANTE	&= USD2.222M/U * 1U/U &\text{Por unidad de tren reemplazada}\\
					& + USD0.111/USD * USD30M/U &\text{Rinde del presupuesto}\\
					& - USD4.444M/Mm * 0.1Mm/U &\text{Cobertura de vías}\\
	LUCROCESANTE	&= USD5.111M/U
\end{align*}
El lucro cesante mínimo por unidad, superior a 5 millones anuales, es mayor al beneficio de 3.8 millones obtenido por el ingreso de una unidad.
Podemos ver entonces que no es redituable incorporar las nuevas formaciones.

\end{enumerate}

\appendix
\includepdf[pages=-,scale=.8,pagecommand={}]{../TP1/7114-20151-Churca-Gamarra-TP1.pdf}
\end{document}
