%% 12pt, a4 paper, with title page
\documentclass[10pt, a4paper, titlepage,
% one-sided
	oneside,
% flush left equations, left equation numbers
	fleqn, leqno]{article}

\usepackage[top=2.5cm, bottom=2cm, left=2.5cm, right=1cm]{geometry}

% Spanish titles
\usepackage[spanish]{babel}

%\usepackage{graphicx}
%\usepackage{amsmath}
%\usepackage{gensymb}
\usepackage{multirow}
\usepackage{rotating}

% dot
%\usepackage{dot2texi}
%\usepackage{tikz}
%\usetikzlibrary{shapes,arrows}

% Font, encoding
\usepackage[utf8]{inputenc}
\usepackage[T1]{fontenc}
\usepackage{times}                % Otros: palatino, bookman, times

% Interactive index
\usepackage{hyperref}

% Attachments
\usepackage{attachfile}


% Make lists without bullets
%\renewenvironment{itemize}{
% \begin{list}{}{
%  \setlength{\leftmargin}{1.5em}
% }
%}{
% \end{list}
%}

% Open sections on right pages
\let\oldsection\section
\def\section{\cleardoublepage\oldsection}

% Code listings
\usepackage{listings}
\lstset{basicstyle=\ttfamily,columns=flexible}

% Watermark
\usepackage{draftwatermark}
\SetWatermarkText{BORRADOR}

% Document properties
\title{71.14 - Modelos y Optimización I - TP1}
\author{
	Federico Churca-Torrusio\\
	Cynthia Gamarra}
\date{2015 - Primer Cuatrimestre\\
	Última modificación: \today}
\hypersetup{
  colorlinks = true,
  urlcolor = red,
  pdflang = es,
  pdfauthor = {Federico Churca-Torrusio, Cynthia Gamarra},
  pdfproducer = {Federico Churca-Torrusio},
  pdfcreator = pdflatex,
  pdftitle = {71.14 - Modelos y Optimización I - TP1},
  pdfsubject = {},
  pdfpagemode = UseNone
}

\begin{document}
\maketitle
\thispagestyle{empty}
\cleardoublepage
%\pagenumbering{roman}
%\setcounter{page}{1}
%\tableofcontents
%\cleardoublepage
%\listoffigures
%\listoftables
%\cleardoublepage

\pagenumbering{arabic}
\setcounter{page}{1}

\section{Parte A}
\section{Parte B}
\subsection{Objetivo}
Determinar un plan de expansión de la red ferroviaria para maximizar los beneficios en el transcurso de un año.

\subsection{Hipótesis}
\begin{itemize}
 \item Las opciones de inversión en trenes y en vías no son absolutamente excluyentes, sino que comparten un presupuesto a distribuir.
 \item El plan es a un año.
 \item La instalación de trenes y vías se completa antes del principio del año a analizar.
 \item Los réditos obtenidos se computan al final del año a analizar.
 \item Se dispone del dinero para la elaboración del plan.
 \item No hay inflación, o en caso de existir, la relación entre los precios de costos se mantienen constantes en el período analizado.
 \item Los costos de mano de obra están incluidos en los gastos de cualquiera de los dos planes a realizar.
 \item Las vías no cubiertas no generan costos.
\end{itemize}

\subsection{Modelo}
\subsubsection{Variables}

\begin{tabular}{|l|r|r|r|} \hline
    & Trenes [u] & Vías [km] & Dinero [USD]\\ \hline
  Cantidad inicial & 1000 & 48K & 50M\\ \hline
  Compra mínima & 40 & 0 & 0\\ \hline
  Costo inicial (USD/u) & 20M & 50K & 0\\ \hline
  Rédito anual (USD/yu) & 2M & 10K* & 0\\ \hline
\end{tabular}

\subsubsection{Planteo Matemático}
Se desea maximizar el siguiente funcional:

MAX     $$     20000000 T + 10000 V$$

\textattachfile{tp1b.lp}{tp1b.lp}
\lstinputlisting{tp1b.lp}

\end{document}
