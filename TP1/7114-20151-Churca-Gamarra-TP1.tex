%% 12pt, a4 paper, with title page
\documentclass[10pt, a4paper, titlepage,
% one-sided
	oneside,
% flush left equations, left equation numbers
	fleqn, leqno]{article}

\usepackage[top=2.5cm, bottom=2cm, left=2.5cm, right=1cm]{geometry}

% Spanish titles
\usepackage[spanish]{babel}

%\usepackage{graphicx}
%\usepackage{amsmath}
%\usepackage{gensymb}
\usepackage{multirow}
\usepackage{rotating}

% dot
%\usepackage{dot2texi}
%\usepackage{tikz}
%\usetikzlibrary{shapes,arrows}

% Font, encoding
\usepackage[utf8]{inputenc}
\usepackage[T1]{fontenc}
\usepackage{times}                % Otros: palatino, bookman, times

% Interactive index
\usepackage{hyperref}

% Attachments
\usepackage{attachfile}


% Make lists without bullets
%\renewenvironment{itemize}{
% \begin{list}{}{
%  \setlength{\leftmargin}{1.5em}
% }
%}{
% \end{list}
%}

% Open sections on right pages
\let\oldsection\section
\def\section{\cleardoublepage\oldsection}

\newcommand{\attachlisting}[1]{\textattachfile{#1}{#1}\lstinputlisting{#1}}

% Code listings
\usepackage{listings}
% Use a monospace font, don't pad with spaces
\lstset{basicstyle=\ttfamily,columns=flexible}

% Watermark
%\usepackage{draftwatermark}
%\SetWatermarkText{BORRADOR}

% Document properties
\title{71.14 - Modelos y Optimización I - TP1}
\author{
	Federico Churca-Torrusio \\\texttt{fchurca@fi.uba.ar}
	\and
	Cynthia Gamarra \\\texttt{cyntgamarra@gmail.com}}
\date{2015 - Primer Cuatrimestre}
%	Última modificación: \today}
\hypersetup{
  colorlinks = true,
  urlcolor = red,
  pdflang = es,
  pdfauthor = {Federico Churca-Torrusio, Cynthia Gamarra},
  pdfproducer = {Federico Churca-Torrusio},
  pdfcreator = pdflatex,
  pdftitle = {71.14 - Modelos y Optimización I - TP1},
  pdfsubject = {},
  pdfpagemode = UseNone
}

\begin{document}
\maketitle
\thispagestyle{empty}
\cleardoublepage
%\pagenumbering{roman}
%\setcounter{page}{1}
%\tableofcontents
%\cleardoublepage
%\listoffigures
%\listoftables
%\cleardoublepage

\pagenumbering{arabic}
\setcounter{page}{1}

\section{Parte A}

\subsection{Análisis del problema}
Dado el nuevo sistema de elecciones llamado las PASO, donde es una primera instancia en la cual no solamente se pelean entre partidos sino también dentro de los partidos, está Frank Underwood, máximo candidato del partido Argentinidad, que quiere conseguir el apoyo de los diferentes sectores de su partido. Sabe que consiguiendo el apoyo de todos los sectores podrá ganar las internas sin perder la unidad del partido en el camino.\\
El problema el cual no estamos enfrentando es de tipo problema combinatorio y se quiere encontrar la mejor combinación óptima.
\subsection{Objetivo}
Determinar la estrategia para conseguir el mayor apoyo de los sectores para ganar las internas, sin perder la unidad del partido en el camino, durante el periódo de las elecciones.

\subsection{Hipótesis}
\begin{itemize}
 \item Una vez que consigue el apoyo de un sector, éste lo apoyará hasta el final de las elecciones.
 \item Las estrategias, como ser show mediático o partido fuertementemente unificado son irrelevantes a la hora de resolver el problema.
\end{itemize}
\subsection{Modelo}
\subsubsection{Variables}

\begin{table}[h!t]
  \centering
  \begin{tabular}{ | c | p{7cm} | }
    \hline
    \textbf{Variables} & \textbf{Descripción} \\ \hline
     \(Xh\)          &  Corriente histórica. \\ \hline
     \(Xj\)          & Juventud. \\ \hline
     \(Xe\)          & Empresarios. \\ \hline
     \(Xs\)          & Sindicatos. \\ \hline
     \(Yc\)          & Huelga en Campana: A favor del sector empresario. \\ \hline
     \(Yd\)          & Lista de diputados: Otorgar vacante a juventud. \\ \hline
     \(Yi\)          & Apertura de importaciones. \\ \hline
     \(Yvh\)          & Vicepresidencia: Corriente Histórica. \\ \hline
     \(Yvh\)          & Vicepresidencia: Juventud. \\ \hline
     \(Yve\)          & Vicepresidencia: Empresarios. \\ \hline
     \(Yvs\)          & Vicepresidencia: Sindicatos. \\ \hline
  \end{tabular}

\end{table}

\subsubsection{Planteo Matemático en LINGO}
\attachlisting{tp1a.lp}

\subsection{Análisis de resultados}
En el informe detallada debajo, se puede observar que  cuando se le otorga la vicepresidencia a los empresarios amigos y la vacante a la lista a la juventud, obtiene el apoyo de todos los sectores.
\subsubsection{Informe de las soluciones de las variables}
\attachlisting{tp1a.out}

\section{Parte B}
\subsection{Análisis del problema}
Frank Underwood debe elegir entre dos opciones: invertir en la red en si o en la compra de trenes.
El tipo de problema que se debe trabajar es un tipo de problema de análisis de actividad donde se quiere maximizar el beneficio total.
\subsection{Objetivo}
Determinar un plan de expansión de la red ferroviaria para maximizar los beneficios en el transcurso de un año.

\subsection{Hipótesis}
\begin{itemize}
 \item Las opciones de inversión en trenes y en vías no son absolutamente excluyentes, sino que comparten un presupuesto a distribuir.
 \item El plan es a un año.
 \item La instalación de trenes y vías se completa antes del principio del año a analizar.
 \item Los réditos obtenidos se computan al final del año a analizar.
 \item Se dispone del dinero para la elaboración del plan.
 \item No hay inflación, o en caso de existir, la relación entre los precios de costos se mantienen constantes en el período analizado.
 \item Los costos de mano de obra están incluidos en los gastos de cualquiera de los dos planes a realizar.
 \item Las vías no cubiertas no generan costos.
\end{itemize}

\subsection{Modelo}
\subsubsection{Variables}

\begin{tabular}{|l|r|r|r|} \hline
    & Trenes [u] & Vías [km] & Dinero [USD]\\ \hline
  Cantidad inicial & 1000 & 48K & 50M\\ \hline
  Compra mínima & 40 & 0 & 0\\ \hline
  Costo inicial (USD/u) & 20M & 50K & 0\\ \hline
  Rédito anual (USD/yu) & 2M & 10K* & 0\\ \hline
\end{tabular}

\subsubsection{Planteo Matemático en LINGO}
\attachlisting{tp1b.lp}

\subsection{Análisis de resultados}
En el informe detallado debajo, se puede observar que el óptimo se da cuando se compra 40 unidades de trenes y se invierte en la creación de 4000 km de vías.  
\subsubsection{Informe de las soluciones de las variables}
\attachlisting{tp1b.out}

\end{document}
